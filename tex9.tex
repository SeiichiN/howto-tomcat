\documentclass[dvipdfmx]{jsarticle}

\title{西暦和暦変換プログラムの作成(Java版)}
\author{Seiichi Nukayama}
\date{2020-05-05}
\usepackage{tcolorbox}
\usepackage{color}
\usepackage{listings, plistings}

% Java
\lstset{% 
  frame=single,
  backgroundcolor={\color[gray]{.9}},
  stringstyle={\ttfamily \color[rgb]{0,0,1}},
  commentstyle={\itshape \color[cmyk]{1,0,1,0}},
  identifierstyle={\ttfamily}, 
  keywordstyle={\ttfamily \color[cmyk]{0,1,0,0}},
  basicstyle={\ttfamily},
  breaklines=true,
  xleftmargin=0zw,
  xrightmargin=0zw,
  framerule=.2pt,
  columns=[l]{fullflexible},
  numbers=left,
  stepnumber=1,
  numberstyle={\scriptsize},
  numbersep=1em,
  language={Java},
  lineskip=-0.5zw,
  morecomment={[s][{\color[cmyk]{1,0,0,0}}]{/**}{*/}},
}
%\usepackage[dvipdfmx]{graphicx}
\usepackage{url}
\usepackage[dvipdfmx]{hyperref}
\usepackage{amsmath, amssymb}
\usepackage{itembkbx}
\usepackage{eclbkbox}	% required for `\breakbox' (yatex added)
\fboxrule=1pt
\parindent=1em
\begin{document}

%% 修正時刻: Tue May  5 10:19:29 2020


\section{セッションスコープ}

\textsf{example} フォルダが増えてきたので、そろそろ別のフォルダを作成して、そこで作業する。

\textsf{example8} というフォルダを作成する。以下のような構成になる。


\begin{tcolorbox}
\begin{verbatim}
./example8
├── WEB-INF
│   ├── classes
│   └── jsp
│       ├── registerConfirm.jsp
│       ├── registerDone.jsp
│       └── registerForm.jsp
├── build.xml
├── index.html
└── src
    ├── model
    │   ├── RegisterUserLogic.java
    │   └── User.java
    └── servlet
        └── RegisterUser.java
\end{verbatim}
\end{tcolorbox}

javaファイル、jspファイルはこれから作成する。

build.xmlは次のようになる。

\begin{tcolorbox}[title=build.xml]
\begin{verbatim}
<?xml version=''1.0'' ?>
<project name=''example8'' default=''compile'' basedir=''.''>
  <target name=''compile''>
      <javac includeAntRuntime=''false''
             encoding=''UTF-8''
             srcdir=''./src''
             destdir=''./WEB-INF/classes''
             />
  </target>
</project>
\end{verbatim}
\end{tcolorbox}

\verb!C:\pleiades\tomcat\9\conf\Catalina\localhost\example8.xml! を作成して、以下の内容とする。

\begin{tcolorbox}[title=example8.xml]
\begin{verbatim}
<?xml version='1.0' encoding='utf-8'?>
<Context path=''/example8'' docBase=''C:\Users\user\Documents\example8'' />
\end{verbatim}
\end{tcolorbox}



\end{document}

%% 修正時刻: Sat May  2 15:10:04 2020


%% 修正時刻: Fri Jul 31 16:43:19 2020
