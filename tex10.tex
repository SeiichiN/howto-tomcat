\documentclass[dvipdfmx]{jsarticle}

\title{西暦和暦変換プログラムの作成(Java版)}
\author{Seiichi Nukayama}
\date{2020-05-05}
\usepackage{tcolorbox}
\usepackage{color}
\usepackage{listings, plistings}

% Java
\lstset{% 
  frame=single,
  backgroundcolor={\color[gray]{.9}},
  stringstyle={\ttfamily \color[rgb]{0,0,1}},
  commentstyle={\itshape \color[cmyk]{1,0,1,0}},
  identifierstyle={\ttfamily}, 
  keywordstyle={\ttfamily \color[cmyk]{0,1,0,0}},
  basicstyle={\ttfamily},
  breaklines=true,
  xleftmargin=0zw,
  xrightmargin=0zw,
  framerule=.2pt,
  columns=[l]{fullflexible},
  numbers=left,
  stepnumber=1,
  numberstyle={\scriptsize},
  numbersep=1em,
  language={Java},
  lineskip=-0.5zw,
  morecomment={[s][{\color[cmyk]{1,0,0,0}}]{/**}{*/}},
}
%\usepackage[dvipdfmx]{graphicx}
\usepackage{url}
\usepackage[dvipdfmx]{hyperref}
\usepackage{amsmath, amssymb}
\usepackage{itembkbx}
\usepackage{eclbkbox}	% required for `\breakbox' (yatex added)
\fboxrule=1pt
\parindent=1em
\begin{document}

%% 修正時刻: Tue May  5 10:19:29 2020


\section{アプリケーションスコープ}

\textsf{sample9}というフォルダにまとめて作成する。

\verb!C:\pleiades\tomcat\9\conf\Catalina\localhost\example9.xml! を作成し、以下の内容とする。

\begin{lstlisting}[caption=example9.xml]
<?xml version="1.0" encoding="utf-8" ?>
<Context path="/example9" docBase="C:\Users\user\Documents\sukkiri\example9" />
\end{lstlisting}

``\textsf{sukkiri}''フォルダに ``\textsf{example9}''フォルダを作成し、その
構成は、以下である。

\begin{tcolorbox}
\begin{verbatim}
./
├── WEB-INF
│   ├── classes
│   └── jsp
│       └── minatoIndex.jsp
├── build.xml
├── index.html
└── src
    ├── model
    │   ├── SiteEV.java
    │   └── siteEVLogic.java
    └── servlet
        └── MinatoIndex.java
\end{verbatim} 
\end{tcolorbox}

build.xmlは以下のとおり。

\begin{lstlisting}[caption=build.xml]
<?xml version="1.0" ?>
<project name="example9" default="compile" basedir=".">
  <target name="compile">
      <javac includeAntRuntime="false"
             encoding="UTF-8"
             srcdir="./src"
             destdir="./WEB-INF/classes"
             />
  </target>
</project>
\end{lstlisting}

p254の''siteEV.java''、p255の''SiteDVLogic.java''、p256の''MinatoIndex.java''、
p257の''minatoIndex.jsp''を入力する。




\include{end}

%% 修正時刻: Sat Aug  1 19:29:32 2020
