\documentclass[dvipdfmx]{jsarticle}

\title{西暦和暦変換プログラムの作成(Java版)}
\author{Seiichi Nukayama}
\date{2020-05-05}
\usepackage{tcolorbox}
\usepackage{color}
\usepackage{listings, plistings}

% Java
\lstset{% 
  frame=single,
  backgroundcolor={\color[gray]{.9}},
  stringstyle={\ttfamily \color[rgb]{0,0,1}},
  commentstyle={\itshape \color[cmyk]{1,0,1,0}},
  identifierstyle={\ttfamily}, 
  keywordstyle={\ttfamily \color[cmyk]{0,1,0,0}},
  basicstyle={\ttfamily},
  breaklines=true,
  xleftmargin=0zw,
  xrightmargin=0zw,
  framerule=.2pt,
  columns=[l]{fullflexible},
  numbers=left,
  stepnumber=1,
  numberstyle={\scriptsize},
  numbersep=1em,
  language={Java},
  lineskip=-0.5zw,
  morecomment={[s][{\color[cmyk]{1,0,0,0}}]{/**}{*/}},
}
%\usepackage[dvipdfmx]{graphicx}
\usepackage{url}
\usepackage[dvipdfmx]{hyperref}
\usepackage{amsmath, amssymb}
\usepackage{itembkbx}
\usepackage{eclbkbox}	% required for `\breakbox' (yatex added)
\fboxrule=1pt
\parindent=1em
\begin{document}

%% 修正時刻: Tue May  5 10:19:29 2020


\section{Webのしくみ}

\subsection{HTTP通信のしくみ}

スタートを右クリック ー アプリと機能 ー プログラムと機能 ー Windowsの機能の有効化または無効化 ー

Windowsの機能のダイアログボックスで
Telnet Client にチェックを入れて OK

ファイルのインストールが始まり、Windowsを再起動する。

MAMPなどで、Apacheを起動しておく。

あるいは、任意のフォルダで、「php -S localhost:80」とすると、そのポートで簡易Webサーバが起動する。

メモ帳などを起動して、以下の文字を入力してコピーしておく。

\begin{tcolorbox}
GET / HTTP/1.1
Host: localhost
\end{tcolorbox}

コマンドプロンプトを起動する。

\verb!> telnet localhost 80 <リターン>!

何も表示されなくなる。

「\verv!Ctrl - ]!」とする。

\begin{tcolorbox}
Microsoft Telnet クライアントへようこそ
エスケープ文字は 'CTRL+]' です
Microsoft Telnet> 
\end{tcolorbox}

と表示される。

\fbox{Microsoft telnet> set localecho <リターン>}

とする。

\begin{tcolorbox}
ローカル エコー: オン
Microsoft Telnet>
\end{tcolorbox}

と表示されるので、

\fbox{<リターン>}

とする。また、何も表示されなくなる。

左上の黒いアイコンを左クリック ー 編集 ー 貼り付け

\begin{tcolorbox}
GET / HTTP/1.1
Host: loalhost
(空行) <リターン>
\end{tcolorbox}

とすると、

\begin{tcolorbox}
HTTP/1.1 200 OK
Date: Thu, 23 Jul 2020 11:15:45 GMT
Connection: close
Content-Type: text/html; charset=UTF-8
Content-Length: 147

<!doctype html>
<html lang="ja">
<head>
<meta charset="utf-8">
<title>Sample</title>
</head>
<body>
<h1>Sample Page</h1>
</body>
</html>
\end{tcolorbox}

というような文字列が表示される。

空行で区切られていて、前段が「ヘッダ部」、後段が「ボディ部」である。


\subsection{開発環境を準備する}

\subsubsection{フォルダの作成}

eclipse ワークディレクトリが C:\pleiades\workspace\sukkiri なので、
それとは別にする。

C:\Users\user\Documents\sukkiri というフォルダを作成。
その中に example というプロジェクト(フォルダ)を作成する。

\subsubsection{Tomcat への登録}

C:\pleiades\tomcat\9 の中に Catalina というフォルダを作成し、その中に localhost というフォルダを作成する。

C:\pleiades\tomcat\9\conf\Catalina\localhost\ に example.xml というファイルを作成し、以下の内容とする。

\begin{tcolorbox}
<?xml version='1.0' encoding='utf-8'?>
<Context path="/example" docBase="C:/Users/user/Documents/sukkiri/example" />
\end{tcolorbox}

\subsection{開発環境を体験する}

\subsubsection{exampleフォルダの内容}

exampleフォルダのツリー構造は以下である。


./example
  ├── WEB-INF
  │   ├── classes
  │   └── lib
  ├── src
  └── index.html

\subsubsection{index.html の作成}

こんにちは、HTML!!

と表示するだけの index.html を作成し、上のツリー図のように配置する。

\begin{lstlisting}
<!doctype html>
<html lang="ja">
<head>
  <meta charset="utf-8">
  <title>すっきり</titie>
</head>
<body>
  <h1>こんにちは、HTML!!</h1>
</body>
</html>
\end{lstlisting}



5) ブラウザで確認

Tomcat を再起動後、http://localhost:8080/example/ にアクセスすることでブラウザに表示できる。

\end{document}

%% 修正時刻: Sat May  2 15:10:04 2020

