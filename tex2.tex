\documentclass[dvipdfmx]{jsarticle}

\title{西暦和暦変換プログラムの作成(Java版)}
\author{Seiichi Nukayama}
\date{2020-05-05}
\usepackage{tcolorbox}
\usepackage{color}
\usepackage{listings, plistings}

% Java
\lstset{% 
  frame=single,
  backgroundcolor={\color[gray]{.9}},
  stringstyle={\ttfamily \color[rgb]{0,0,1}},
  commentstyle={\itshape \color[cmyk]{1,0,1,0}},
  identifierstyle={\ttfamily}, 
  keywordstyle={\ttfamily \color[cmyk]{0,1,0,0}},
  basicstyle={\ttfamily},
  breaklines=true,
  xleftmargin=0zw,
  xrightmargin=0zw,
  framerule=.2pt,
  columns=[l]{fullflexible},
  numbers=left,
  stepnumber=1,
  numberstyle={\scriptsize},
  numbersep=1em,
  language={Java},
  lineskip=-0.5zw,
  morecomment={[s][{\color[cmyk]{1,0,0,0}}]{/**}{*/}},
}
%\usepackage[dvipdfmx]{graphicx}
\usepackage{url}
\usepackage[dvipdfmx]{hyperref}
\usepackage{amsmath, amssymb}
\usepackage{itembkbx}
\usepackage{eclbkbox}	% required for `\breakbox' (yatex added)
\fboxrule=1pt
\parindent=1em
\begin{document}

%% 修正時刻: Tue May  5 10:19:29 2020


\section{Webのしくみ}

\subsection{HTTP通信のしくみ}

スタートを右クリック ー アプリと機能 ー プログラムと機能 ー Windowsの機能の有効化または無効化 ー

Windowsの機能のダイアログボックスで
\textbf{Telnet Client} にチェックを入れて \textbf{OK}

ファイルのインストールが始まる。終了したら、Windowsを再起動する。

MAMPなどで、Apacheを起動しておく。

あるいは、任意のフォルダで、「php -S localhost:80」とすると、そのポートで簡易Webサーバが起動する。

メモ帳などを起動して、以下の文字を入力してコピーしておく。

\begin{tcolorbox}
GET / HTTP/1.1 \\
Host: localhost
\end{tcolorbox}

コマンドプロンプトを起動する。

\begin{tcolorbox}
\verb! > telnet localhost 80 <リターン> !
\end{tcolorbox}

何も表示されなくなる。

\fbox{Ctrl - ]} とする。

\begin{tcolorbox}
Microsoft Telnet クライアントへようこそ \\
エスケープ文字は 'CTRL+]' です \\
Microsoft Telnet\textgreater
\end{tcolorbox}

と表示される。

\fbox{Microsoft telnet\textgreater  set localecho \textless リターン\textgreater}

とする。

\begin{tcolorbox}
ローカル エコー: オン \\
Microsoft Telnet>
\end{tcolorbox}

と表示されるので、

\fbox{\textless リターン\textgreater}

とする。また、何も表示されなくなる。

左上の黒いアイコンを左クリック ー 編集 ー 貼り付け

\begin{tcolorbox}
GET / HTTP/1.1 \\
Host: loalhost \\
(空行) \textless リターン\textgreater
\end{tcolorbox}

とすると、

 \begin{tcolorbox}
  \begin{verbatim}
HTTP/1.1 200 OK
Date: Thu, 23 Jul 2020 11:15:45 GMT
Connection: close
Content-Type: text/html; charset=UTF-8
Content-Length: 147

<!doctype html>
<html lang="ja">
<head>
<meta charset="utf-8">
<title>Sample</title>
</head>
<body>
<h1>Sample Page</h1>
</body>
</html>
\end{verbatim}
 \end{tcolorbox}

というような文字列が表示される。

空行で区切られていて、前段が「ヘッダ部」、後段が「ボディ部」である。


\subsection{開発環境を準備する}

\subsubsection{フォルダの作成}

Webアプリの場所は \textbf{webapp} フォルダだけでなく、任意のフォルダを作成して、それを
Tomcat に登録することもできる。

eclipse ワークディレクトリが \verb!C:\pleiades\workspace\sukkiri! なので、
それとは別の場所に作成することにする。

\verb!C:\Users\user\Documents\sukkiri! というフォルダを作成。
その中に \textbf{example} というプロジェクト(フォルダ)を作成する。

\subsubsection{Tomcat への登録}

\verb!C:\pleiades\tomcat\9! の中に \textbf{Catalina} というフォルダを作成し、
その中に \textbf{localhost} というフォルダを作成する。

\verb!C:\pleiades\tomcat\9\conf\Catalina\localhost\! に \textbf{example.xml} というファイルを作成し、以下の内容とする。

\begin{tcolorbox}
\begin{verbatim}
<?xml version='1.0' encoding='utf-8'?>
<Context path="/example" docBase="C:/Users/user/Documents/sukkiri/example" />
\end{verbatim}
\end{tcolorbox}

\subsection{開発環境を体験する}

\subsubsection{exampleフォルダの内容}

exampleフォルダのツリー構造は以下である。

\begin{tcolorbox}
\begin{verbatim}
./example
  ├── WEB-INF
  │   ├── classes
  │   └── lib
  ├── src
  └── index.html
\end{verbatim}
\end{tcolorbox}

WEB-INFフォルダに配置したファイルは、ブラウザのURL欄に指定しても見ることはできない。
サーブレットやJSPから呼び出せるだけである。
サーブレットはこのフォルダのclassesフォルダの中に置く。

それに対して、プロジェクトフォルダ直下においたファイル(この例ではindex.html)は、
ブラウザのURL欄に指定して見ることができる。JSPファイルもここに置くことができる。
JSPファイルでも、ブラウザのURL欄で指定して見ることができないようにするために、
WEB-INFフォルダの中に置くことができる。その場合、\textbf{jsp}というフォルダを
作成してその中に置くことが多い。

\textbf{src} フォルダには、サーブレットやクラスファイルのソースコードを置く。
コンパイルした classファイルを WEB-INF/classes フォルダの中に置く。

\subsubsection{index.html の作成}

こんにちは、HTML!!

と表示するだけの index.html を作成し、上のツリー図のように配置する。

\begin{lstlisting}
<!doctype html>
<html lang="ja">
<head>
  <meta charset="utf-8">
  <title>すっきり</titie>
</head>
<body>
  <h1>こんにちは、HTML!!</h1>
</body>
</html>
\end{lstlisting}



\subsubsection{ブラウザで確認}

Tomcat を再起動後、http://localhost:8080/example/ にアクセスすることでブラウザに表示できる。

\end{document}

%% 修正時刻: Sat May  2 15:10:04 2020


%% 修正時刻: Fri Jul 24 09:36:20 2020
