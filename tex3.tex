\documentclass[dvipdfmx]{jsarticle}

\title{西暦和暦変換プログラムの作成(Java版)}
\author{Seiichi Nukayama}
\date{2020-05-05}
\usepackage{tcolorbox}
\usepackage{color}
\usepackage{listings, plistings}

% Java
\lstset{% 
  frame=single,
  backgroundcolor={\color[gray]{.9}},
  stringstyle={\ttfamily \color[rgb]{0,0,1}},
  commentstyle={\itshape \color[cmyk]{1,0,1,0}},
  identifierstyle={\ttfamily}, 
  keywordstyle={\ttfamily \color[cmyk]{0,1,0,0}},
  basicstyle={\ttfamily},
  breaklines=true,
  xleftmargin=0zw,
  xrightmargin=0zw,
  framerule=.2pt,
  columns=[l]{fullflexible},
  numbers=left,
  stepnumber=1,
  numberstyle={\scriptsize},
  numbersep=1em,
  language={Java},
  lineskip=-0.5zw,
  morecomment={[s][{\color[cmyk]{1,0,0,0}}]{/**}{*/}},
}
%\usepackage[dvipdfmx]{graphicx}
\usepackage{url}
\usepackage[dvipdfmx]{hyperref}
\usepackage{amsmath, amssymb}
\usepackage{itembkbx}
\usepackage{eclbkbox}	% required for `\breakbox' (yatex added)
\fboxrule=1pt
\parindent=1em
\begin{document}

%% 修正時刻: Tue May  5 10:19:29 2020


\section{サーブレットの基礎と作成方法}

作成するサーブレットのプロジェクト名は、\textbf{example}。

\vspace{3mm}
参考URL
\vspace{-3mm}
\begin{itemize}
 \item \href{サーブレットクラスの作成}{https://sukkiri.jp/technologies/ides/eclipse/servlet\_create.html?ssj=3}
 \item \href{サーブレットの実行}{https://sukkiri.jp/technologies/ides/eclipse/servlet\_exec.html?ssj=3}
\end{itemize}

ドキュメントフォルダに \textbf{sukkiri} フォルダがあるはずだから、
その中に \textbf{example} フォルダを作成する。

今回は、前回で作成した exampleフォルダがすでにあるので、それを使うことにする。

フォルダの構造は以下のとおり。

\begin{tcolorbox}
\begin{verbatim}
./example
  ├── WEB-INF
  │   ├── classes
  │   └── lib
  ├── src
  └── index.html
\end{verbatim}
\end{tcolorbox}

「スッキリ」本には index.html は無いが、あると便利なので作っておく。
内容は、以下とする。前回作成したものを少し修正する。

\begin{lstlisting}
 <!doctype html>
 <html lang=''ja''>
 <head>
 <meta charset=''utf-8''>
 <title>SERVLET</title>
 </head>
 <body>
 <h1>sukkiri-chap3</h1>
 <p><a href=''#''>SampleServlet</a></p>
 </body>
 </html>
\end{lstlisting}


\subsection{サーブレットの作成}

\textbf{src}フォルダに \textbf{servlet} フォルダを作成し、
そこに \textbf{SampleServlet.java} を作成する。

\textbf{servlet} フォルダがパッケージ名に相当する。

\begin{tcolorbox}
\begin{verbatim}
./example
├── WEB-INF
│   ├── classes
│   └── lib
├── index.html
└── src
    └── servlet
        └── SampleServlet.java
\end{verbatim}
\end{tcolorbox}

「本」のp83からp92までの内容を実際にやってみる。

以下の内容を好みのエディタで入力する。

\begin{lstlisting}[caption=SampleServlet.java]
package servlet;

import java.io.IOException;
import java.io.PrintWriter;                                 // <1>

import javax.servlet.ServletException;
import javax.servlet.annotation.WebServlet;                 // <2>
import javax.servlet.http.HttpServlet;
import javax.servlet.http.HttpServletRequest;
import javax.servlet.http.HttpServletResponse;

@WebServlet("/SampleServlet")
public class SampleServlet extends HttpServlet {
    private static final long serialVersionUID = 1L;

    protected void doGet( HttpServletRequest request,
                          HttpServletResponse response )
        throws ServletException, IOException {

        response.setContentType("text/html; charset=UTF-8");
        PrintWriter out = response.getWriter();
        out.println("<html>");
        out.println("---");
        out.println("</html>");
    }
}
\end{lstlisting}

\begin{verbatim}
    <1> -- PrintWriter で必要。
    <2> -- @WebServer("/SampleServlet")で必要。
\end{verbatim}

次に、コンパイルを行う。srcフォルダにいるとする。

\fbox{\textgreater  javac  -d  ../WEB-INF/classes  servlet/*.java}

\verb!-d! オプションは コンパイルしたあとにできる classファイルをどこに置くかという指定である。
Tomcatの場合、WEB-INF/classes に置くことに決まっている。

コンパイルするソースファイルは servlet/*.java と指定している。servletフォルダの中のすべての
javaファイルという意味である。

コンパイルしようとすると エンコードに関するエラーがずらずらと表示されるかもしれない。
UTF-8の文字コードで作成したファイルを Windowsのコマンドプロンプトではそのままでは
コンパイルできない。\\
以下のように \verb!-encoding UTF-8! とオプションをつけてコンパイルする。

\fbox{\textgreater  javac  -encoding  UTF-8  -d  ../WEB-INF/classes  servlet/*.java}


サーブレットはコンパイルするごとに Tomcat を再起動しなければならない。
Tomcat が動作しているコマンドプロンプトの画面を閉じるか、Ctrl + C として画面を閉じて、
再度 Tomcat を起動するショートカットをダブルクリックして Tomcat を起動する。

そして、ブラウザでアクセス。

URLは、\verb!http://localhost:8080/example/SampleServlet! である。

index.html を開いて、以下の箇所にURL文字列を入れる。

\begin{tcolorbox}
\verb! <p><a href=''/example/SampleServlet''>SampleServlet</a></p> !
\end{tcolorbox}

ブラウザに \verb! http://localhost:8080/example/ ! とすると、index.html が表示されるので、
リンク先をクリックしてうまく表示されると成功である。

もっとも、「---」と表示されるだけであるが。


\subsection{サーブレットクラスを作成して実行する}

p93からの内容は、以下である。

SampleServlet.java を以下のように修正する。

\begin{lstlisting}[caption=SampleServlet.java]
package servlet;

import java.io.IOException;
import java.io.PrintWriter;
import java.text.SimpleDateFormat;
import java.util.Date;

import javax.servlet.ServletException;
import javax.servlet.annotation.WebServlet;
import javax.servlet.http.HttpServlet;
import javax.servlet.http.HttpServletRequest;
import javax.servlet.http.HttpServletResponse;

@WebServlet("/SampleServlet")
public class SampleServlet extends HttpServlet {
    private static final long serialVersionUID = 1L;

    protected void doGet( HttpServletRequest request,
                          HttpServletResponse response )
        throws ServletException, IOException {

        // 運勢をランダムで決定
        String [] luckArray = { "超スッキリ", "スッキリ", "最悪" };
        int index = (int) (Math.random() * 3);
        String luck = luckArray[index];

        // 実行日を取得
        Date date = new Date();
        SimpleDateFormat sdf = new SimpleDateFormat("MM月dd日");
        String today = sdf.format(date);

        // HTMLを出力
        response.setContentType("text/html; charset=UTF-8");
        PrintWriter out = response.getWriter();
        out.println("<html>");
        out.println("<head>");
        out.println("<title>スッキリ占い</title>");
        out.println("</head>");
        out.println("<body>");
        out.println("<p>" + today + "の運勢は「" + luck + "」です</p>");
        out.println("</body>");
        out.println("</html>");
    }
}
\end{lstlisting}

Tomcatサーバーを再起動する。

ブラウザでアクセスする。

\verb!http://localhost:8080/example/SampleServlet!




\end{document}

%% 修正時刻: Sat May  2 15:10:04 2020


%% 修正時刻: Fri Jul 24 13:21:18 2020
