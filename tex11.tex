\documentclass[dvipdfmx]{jsarticle}

\title{西暦和暦変換プログラムの作成(Java版)}
\author{Seiichi Nukayama}
\date{2020-05-05}
\usepackage{tcolorbox}
\usepackage{color}
\usepackage{listings, plistings}

% Java
\lstset{% 
  frame=single,
  backgroundcolor={\color[gray]{.9}},
  stringstyle={\ttfamily \color[rgb]{0,0,1}},
  commentstyle={\itshape \color[cmyk]{1,0,1,0}},
  identifierstyle={\ttfamily}, 
  keywordstyle={\ttfamily \color[cmyk]{0,1,0,0}},
  basicstyle={\ttfamily},
  breaklines=true,
  xleftmargin=0zw,
  xrightmargin=0zw,
  framerule=.2pt,
  columns=[l]{fullflexible},
  numbers=left,
  stepnumber=1,
  numberstyle={\scriptsize},
  numbersep=1em,
  language={Java},
  lineskip=-0.5zw,
  morecomment={[s][{\color[cmyk]{1,0,0,0}}]{/**}{*/}},
}
%\usepackage[dvipdfmx]{graphicx}
\usepackage{url}
\usepackage[dvipdfmx]{hyperref}
\usepackage{amsmath, amssymb}
\usepackage{itembkbx}
\usepackage{eclbkbox}	% required for `\breakbox' (yatex added)
\fboxrule=1pt
\parindent=1em
\begin{document}

%% 修正時刻: Tue May  5 10:19:29 2020


\section{アプリケーション作成}

\subsection{準備作業}

``sukkiri''フォルダの中に \textsf{docoTsubu}フォルダを作成する。
そして、その構成を以下とする。

\begin{tcolorbox}
\begin{verbatim}
./docoTsubu
├── WEB-INF/
│   └── jsp/
├── build.xml
└── src/
    ├── model/
    └── servlet/
\end{verbatim} 
\end{tcolorbox}

build.xml は、今までのものをコピーしてここに置く。

このフォルダをTomcatに登録する。

docoTsubuフォルダが以下にあるとする。

\begin{tcolorbox}
\verb! C:\Users\user\Documents\sukkiri\docoTsubu !
\end{tcolorbox}

\verb! C:\pleiades\tomcat\9\conf\Catalina\localhost ! に
\textsf{ docoTsubu.xml } を作成し、以下の内容とする。

\begin{lstlisting}[caption=docoTsubu.xml]
<?xml version='1.0' encoding='utf-8'?>
<Context path="/docoTsubu" docBase="C:\Users\user\Documents\sukkiri\docoTsubu" />
\end{lstlisting}

docoTsubuフォルダに \textsf{index.html} を作成する。

\begin{lstlisting}[caption=index.html]
 <!doctype html>
 <html lang="ja">
 <head>
   <meta charset="utf-8">
   <title>docoTsubu</title>
 </head>
 <body>
   <h1>docoTsubu</h1>
 </body>
 </html>
\end{lstlisting}

デスクトップにある Tomcat のショートカットをダブルクリックして、Tomcatを起動する。

\fbox{ http://localhost:8080/docoTsubu } として \textsf{docoTsubu} のタイトルが
表示されたら成功。


\subsection{プログラムの入力}

p271 からのプログラムを入力する。

p305 までのプログラムリストを入力すると、以下のようなフォルダ構成になっているはず。
(index.html は削除しておく)

\begin{tcolorbox}
\begin{verbatim}
./docoTsubu
├── WEB-INF
│   └── jsp
│       ├── loginResult.jsp
│       ├── logout.jsp
│       └── main.jsp
├── build.xml
├── common.jsp
├── docoTsubu.xml
├── footer.jsp
├── index.jsp
└── src
    ├── model
    │   ├── LoginLogic.java
    │   ├── Mutter.java
    │   ├── PostMutterLogic.java
    │   └── User.java
    └── servlet
        ├── Login.java
        ├── Logout.java
        └── Main.java
\end{verbatim} 
\end{tcolorbox}



\include{end}

%% 修正時刻: Sun Aug  2 09:37:48 2020
