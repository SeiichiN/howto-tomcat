\documentclass[dvipdfmx]{jsarticle}

\title{西暦和暦変換プログラムの作成(Java版)}
\author{Seiichi Nukayama}
\date{2020-05-05}
\usepackage{tcolorbox}
\usepackage{color}
\usepackage{listings, plistings}

% Java
\lstset{% 
  frame=single,
  backgroundcolor={\color[gray]{.9}},
  stringstyle={\ttfamily \color[rgb]{0,0,1}},
  commentstyle={\itshape \color[cmyk]{1,0,1,0}},
  identifierstyle={\ttfamily}, 
  keywordstyle={\ttfamily \color[cmyk]{0,1,0,0}},
  basicstyle={\ttfamily},
  breaklines=true,
  xleftmargin=0zw,
  xrightmargin=0zw,
  framerule=.2pt,
  columns=[l]{fullflexible},
  numbers=left,
  stepnumber=1,
  numberstyle={\scriptsize},
  numbersep=1em,
  language={Java},
  lineskip=-0.5zw,
  morecomment={[s][{\color[cmyk]{1,0,0,0}}]{/**}{*/}},
}
%\usepackage[dvipdfmx]{graphicx}
\usepackage{url}
\usepackage[dvipdfmx]{hyperref}
\usepackage{amsmath, amssymb}
\usepackage{itembkbx}
\usepackage{eclbkbox}	% required for `\breakbox' (yatex added)
\fboxrule=1pt
\parindent=1em
\begin{document}

%% 修正時刻: Tue May  5 10:19:29 2020


\section{Antの利用}

前回は Servlet の作成と実行をおこなった。
そして、コンパイルをおこなったが、コンパイルオプションがたくさんあって、
コンパイルする度に入力するのが面倒である。

そこで、ビルド自動化ツールである ''Ant'' を利用する。

\subsection{Antのインストール}

Antは以下のWebページから入手できる。

\href{THE APACHE ANT PROJECT}{https://ant.apache.org/index.html}

表示されたページの左のメニューから \textbf{Download - Binary Distributions} を
選択する。

表示されたページの下のほうに、以下のバージョンが掲載されている。

\begin{tcolorbox}
\textbf{1.9.15 release - requires minimum of Java 5 at runtime} \\
\textbf{1.10.8 release - requires minimum of Java 8 at runtime}
\end{tcolorbox}

今回使用している Java のバージョンは \textbf{11} なので、\textbf{1.10.8} のほうを
選択する。

その下に4種類の圧縮形式が並べられてある。

zipファイルである \textbf{apache-ant-1.10.8-bin.zip} をクリックする。
するとダウンロードが始まる。

ダウンロードフォルダに保存されるので、エクスプローラからダウンロードフォルダを表示し、
\textbf{apache-ant-1.10.8-bin.zip} を右クリックして、メニューの中から、\textbf{7-zip} を選択し、
その横に表示されたメニューから ''ここに展開'' を選択する。

ダウンロードフォルダに \textbf{apache-ant-1.10.8} フォルダができるので、そのフォルダを
\verb!C:\! のトップに持っていくことにする。

スタート(右クリック) ー システム ー システム情報 ー システムの詳細設定 ー 環境変数
 ー システム環境変数に以下を追加する。

\begin{tcolorbox}
 \textbf{ANT\_HOME} -- \verb!C:\apache-ant-1.10.8! \\
 \textbf{Path} -- \verb!%ANT_HOME%\bin\!
\end{tcolorbox}


\subsection{ビルド設定ファイル}

現在 example フォルダで作業しているので、エクスプローラでそのフォルダを表示する。
そのフォルダの中のトップに \textbf{build.xml} というファイルを作成する。

\begin{tcolorbox}
\begin{verbatim}
./example
├── WEB-INF
│   ├── classes
│   └── lib
├── index.html
├── build.xml
└── src
    └── servlet
        └── SampleServlet.java
\end{verbatim}
\end{tcolorbox}

以下の内容にする。

\begin{lstlisting}[caption=build.xml]
<?xml version="1.0" ?>
<project name="example" default="compile" basedir=".">
  <target name="compile">
    <javac encoding="UTF-8"
           srcdir="./src"
           destdir="./WEB-INF/classes" />
  </target>
</project>
\end{lstlisting}

\verb!warning: 'includeantruntime' was not set, ....! という Warning が出力されるが、\textbf{BUILD SUCCESSFULL} というメッセージが出るはずである。

この warning の出力がわずらわしいので、出ないようにする。

build.xml の内容に以下のように追加する。

\begin{lstlisting}[caption=build.xml]
<?xml version="1.0" ?>
<project name="example" default="compile" basedir=".">
  <target name="compile">
    <javac encoding="UTF-8"
           includeAntRuntime="false"               <!--  これを追加  -->
           srcdir="./src"
           destdir="./WEB-INF/classes" />
  </target>
</project>
\end{lstlisting}

こうすると、warning は出力されない。

includeAntRuntimeオプションは、Antのライブラリを含めるかというスイッチなので、これを
falseにしてもかまわないのだが、ただ、システム環境変数の classpath も参照しなくなる
ので、サーブレットをコンパイルするのに必要な jarファイルも参照しなくなる。
そのための設定を追加する。

\begin{lstlisting}[caption=build.xml]
<?xml version="1.0" ?>
<project name="example" default="compile" basedir=".">
  <target name="compile">
    <javac encoding="UTF-8"
           includeAntRuntime="false"
           srcdir="./src"
           destdir="./WEB-INF/classes"
           classpath="C:/pleiades/tomcat/9/lib/servlet-api.jar;
                      C:/pleiades/tomcat/9/lib/jsp-api.jar"
           />
  </target>
</project>
\end{lstlisting}

こののち、exampeフォルダのトップにて、

\fbox{ \textgreater ant }

とするだけでコンパイルがおこなわれる。

必要なのは、プロジェクトに合わせて \textbf{project name=} のところを書き直すだけである。



\end{document}

%% 修正時刻: Sat May  2 15:10:04 2020


%% 修正時刻: Fri Jul 23 15:07:11 2020
