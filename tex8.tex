\documentclass[dvipdfmx]{jsarticle}

\title{西暦和暦変換プログラムの作成(Java版)}
\author{Seiichi Nukayama}
\date{2020-05-05}
\usepackage{tcolorbox}
\usepackage{color}
\usepackage{listings, plistings}

% Java
\lstset{% 
  frame=single,
  backgroundcolor={\color[gray]{.9}},
  stringstyle={\ttfamily \color[rgb]{0,0,1}},
  commentstyle={\itshape \color[cmyk]{1,0,1,0}},
  identifierstyle={\ttfamily}, 
  keywordstyle={\ttfamily \color[cmyk]{0,1,0,0}},
  basicstyle={\ttfamily},
  breaklines=true,
  xleftmargin=0zw,
  xrightmargin=0zw,
  framerule=.2pt,
  columns=[l]{fullflexible},
  numbers=left,
  stepnumber=1,
  numberstyle={\scriptsize},
  numbersep=1em,
  language={Java},
  lineskip=-0.5zw,
  morecomment={[s][{\color[cmyk]{1,0,0,0}}]{/**}{*/}},
}
%\usepackage[dvipdfmx]{graphicx}
\usepackage{url}
\usepackage[dvipdfmx]{hyperref}
\usepackage{amsmath, amssymb}
\usepackage{itembkbx}
\usepackage{eclbkbox}	% required for `\breakbox' (yatex added)
\fboxrule=1pt
\parindent=1em
\begin{document}

%% 修正時刻: Tue May  5 10:19:29 2020


\section{リクエストスコープ}

\subsection{p192のプログラムコードの入力}

\textsf{C:/pleiades/tomcat/9/conf/Catalina/localhost/example.xml} を使う。
\footnote{servlet フォルダには他にも javaファイルがあるが、ここでは今回入力するファイルのみ表示した。}

\begin{tcolorbox}
\begin{verbatim}
./example
├── WEB-INF
│   ├── classes
│   └── jsp
├── index.html
├── build.xml
├── sample.jsp
├── formSample.jsp
└── src
    └── model
        └── Human.java
\end{verbatim}
\end{tcolorbox}


\subsection{p203からのプログラムコードの入力}

\begin{tcolorbox}
\begin{verbatim}
./example
├── WEB-INF
│   ├── classes
│   └── jsp
│        ├── healthCheck.jsp
│        └── healthCheckResult.jsp
├── index.html
├── build.xml
└── src
     ├── model
     │   ├── Health.java
     │   └── HealthCheckLogic.java
     └── servlet
         └── HealthCheck.java
\end{verbatim}
\end{tcolorbox}





\include{end}

%% 修正時刻: Sat May  2 16:26:44 2020
