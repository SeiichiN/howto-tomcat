\documentclass[dvipdfmx]{jsarticle}

\title{西暦和暦変換プログラムの作成(Java版)}
\author{Seiichi Nukayama}
\date{2020-05-05}
\usepackage{tcolorbox}
\usepackage{color}
\usepackage{listings, plistings}

% Java
\lstset{% 
  frame=single,
  backgroundcolor={\color[gray]{.9}},
  stringstyle={\ttfamily \color[rgb]{0,0,1}},
  commentstyle={\itshape \color[cmyk]{1,0,1,0}},
  identifierstyle={\ttfamily}, 
  keywordstyle={\ttfamily \color[cmyk]{0,1,0,0}},
  basicstyle={\ttfamily},
  breaklines=true,
  xleftmargin=0zw,
  xrightmargin=0zw,
  framerule=.2pt,
  columns=[l]{fullflexible},
  numbers=left,
  stepnumber=1,
  numberstyle={\scriptsize},
  numbersep=1em,
  language={Java},
  lineskip=-0.5zw,
  morecomment={[s][{\color[cmyk]{1,0,0,0}}]{/**}{*/}},
}
%\usepackage[dvipdfmx]{graphicx}
\usepackage{url}
\usepackage[dvipdfmx]{hyperref}
\usepackage{amsmath, amssymb}
\usepackage{itembkbx}
\usepackage{eclbkbox}	% required for `\breakbox' (yatex added)
\fboxrule=1pt
\parindent=1em
\begin{document}

%% 修正時刻: Tue May  5 10:19:29 2020


Eclipse を学ぶためには、まずコマンドラインでTomcatでのアプリ開発に慣れることが必要だと
思う。コマンドラインで作業できれば、Eclipseが何をやってくれているのかがわかるようになる。
僕がJavaを学び始めたころ、「すっきりわかる」シリーズはわかりやすくてよかったのだけれど、
この「サーブレット\&JSP入門」はわかりにくかった。本に書かれているとおり入力すると
Eclipse内でTomcatサーバーを起動させてブラウザから確認できるのだけれど、Eclipseが何を
やっているのかが見えないから、Tomcatサーバーに対して自分が何をやっているのかが
わからない。だから、本を読んでプログラムを動かせても、自分の力がついた感覚がまるで
無かった。

で、試しにコマンドラインで作業をやってみたところ、Eclipseでやっていることが何となく
わかったように思えた。ネットで調べたりしてやってみたんだけれど、Tomcatについての
理解を深めることができたと思う。

このことは実は Dreamweaver の経験からきている。Web制作の初心者がいきなり Dreamweaver
を使ってWebサイトを作っても、めちゃくちゃなサイトができるだけである。何しろ HTMLCSS
の理解なしにやってしまうからである。また、Dreamwerverがやっていることも把握ができない。
Dreamweaverには超便利な機能がいっぱいついているんだけれど、そのありがたみも、
Dreamweaverなしでやることができるからわかる。逆に言うと、自分のHTMLCSSの理解の範囲で
しかDreamweaverを動かすことができないのである。そして、このことは Eclipse についても
言えると思っている。

 \section{準備作業}

 pleiadesのインストールと設定については、\href{https://sukkiri.jp/technologies/ides/eclipse/pleiades_install.html}{Pleiades(Java環境)インストール手順} に説明がある。

\subsection{pleiadesのインストール}

pleiades の all-in-one をインストールする。

\href{https://mergedoc.osdn.jp/}{Pleiades All in One ダウンロード}

pleiadesは「\verb!C:\pleiades!」にインストールしたとする。

\subsection{環境変数の登録}

「スタート」を右クリック - 「システム」- 「システム情報」 - 「システムの詳細設定」
ー 「環境変数」

下の「システムの環境変数」に以下を確認あるいは登録する。

\begin{tcolorbox}
JAVA\_HOME... \verb!C:\pleiades\Java\11! \\
CATALINE\_HOME... \verb!C:\pleiades\tomcat\9! \\
CLASSPATH ... \verb!%CATALINA_HOME%\lib\jsp-api.jar!  \\
\hspace{25mm}  \verb!%CATALINA_HOME%\lib\servlet-api.jar! 
\end{tcolorbox}

Java は 11 を使う。

Tomcat は 9 を使う。

\subsection{Tomcatの起動}

「\verb!C:\pleiades\tomcat\9\bin!」の中に「startup.bat」があるので、
それのショートカットをデスクトップにつくる。\\
(startup.batを右クリック - 送る - デスクトップにショートカットを作成)

デスクトップに作成した startup.batのショートカットをダブルクリックすると、
黒い画面が現れて、文字化けした文字列がだらだらと表示される。
これで Tomcat は起動できている。

Ctrl-C を押すか、黒い画面を閉じれば Tomcat を停止できる。

\subsection{Webページを配置する}

「\verb!C:\pleiades\tomcat\9\webapp!」フォルダに、たとえば「hello」フォルダを作成する。

その中に以下のように index.html を配置する。

\begin{lstlisting}
 <!doctype html>
 <html lang=''ja''>
 <head>
   <meta charset=''utf-8''>
   <title>HELLO</title>
 </head>
 <body>
   <h1>Hello</h1>
 </body>
 </html>
\end{lstlisting}

ブラウザを起動して、\fbox{http://localhost:8080/hello/}にアクセスすると、
index.htmlが表示される。

公開したい Webページをフォルダごと配置すれば、公開できる。

また、webapp フォルダ以外のフォルダを指定することもできる。

\end{document}

%% 修正時刻: Sat May  2 15:10:04 2020


%% 修正時刻: Fri Jul 24 10:03:10 2020
